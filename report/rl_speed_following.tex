\documentclass[11pt]{article}

% --- Packages ---
\usepackage[utf8]{inputenc}
\usepackage[T1]{fontenc}
\usepackage{lmodern}
\usepackage{geometry}
\usepackage{graphicx}
\usepackage{amsmath, amssymb}
\usepackage{booktabs}
\usepackage{caption}
\usepackage{subcaption}
\usepackage{hyperref}
\usepackage{fancyhdr}
\usepackage{siunitx}   % for numbers/units in tables
\usepackage{xcolor}
\usepackage{listings}  % code listings (minted avoided for portability)

% --- Page Setup ---
\geometry{
  a4paper,
  margin=1in
}

% % Math macros for metrics
% \newcommand{\MAE}{\mathrm{MAE}}
% \newcommand{\MSE}{\mathrm{MSE}}
% \newcommand{\RMSE}{\mathrm{RMSE}}

% --- Metadata ---
\title{RL Speed Following Assignment Report}
\author{Ian Wallace}
\date{\today}

\begin{document}

\maketitle

\section{Model and Hyperparameter Modifications}
\subsection{Search Space}
We tune learning rate, batch size, buffer size, entropy/coefficient (where applicable), target noise (TD3/DDPG), and gamma/GAE (PPO).
\begin{table}[h]
  \centering
  \caption{Hyperparameter grid / ranges explored.}
  \label{tab:hparams}
  \begin{tabular}{|l|c|c|c|}
    \hline
    \textbf{Parameter} & \textbf{Values Tested} & \textbf{Best (Val)} & \textbf{Notes} \\
    \hline
    Learning rate & \num{1e-4}, \num{3e-4}, \num{1e-3} & \num{3e-4} & \\
    Batch size    & 128, 256, 512                   & 256            & \\
    Buffer size   & 1e5, 5e5, 1e6                   & 5e5            & \\
    Entropy coeff (SAC) & 0.05, 0.1, auto           & auto           & \\
    PPO clip      & 0.1, 0.2                        & 0.2            & \\
    TD3 noise std & 0.1, 0.2                        & 0.1            & \\
    \hline
  \end{tabular}
\end{table}

\subsection{Implementation Notes}
Briefly describe any architectural or configuration changes (e.g., network width/depth, activation functions).


\section{Episode Length Variations}
We modify the dataloader to accept an episode length $L$ as a runtime parameter. We test $L \in \{50, 100, 200\}$ and report how $L$ influences stability, sample efficiency, and final error.

\begin{table}[h]
  \centering
  \caption{Effect of episode length on performance (illustrative placeholders).}
  \label{tab:chunks}
  \begin{tabular}{lcccc}
    \hline
    \textbf{Algo} & \textbf{$L$} & \textbf{Final MAE} & \textbf{Convergence Steps} & \textbf{Notes}\\
    \hline
    SAC  & 50  & 0.XXX & NNNK & \\
    SAC  & 100 & 0.XXX & NNNK & \\
    SAC  & 200 & 0.XXX & NNNK & \\
    PPO  & 50  & 0.XXX & NNNK & \\
    \hline
  \end{tabular}
\end{table}

% \begin{figure}[h!]
%   \centering
%   \includegraphics[width=0.7\textwidth]{example-image}
%   \caption{Example figure caption.}
%   \label{fig:example}
% \end{figure}

% \begin{table}[h!]
%   \centering
%   \begin{tabular}{lcc}
%     \textbf{Item} & \textbf{Value 1} & \textbf{Value 2} \\
%     A & 10 & 20 \\
%     B & 15 & 25 \\ \hline
%   \end{tabular}
%   \caption{Example table.}
%   \label{tab:example}
% \end{table}

\section{Conclusion}
Summarize your findings and future directions.

\end{document}